\section{Introduction}

Phantom lines between the engineering of structures through scientific intellect and the designing of structures through artistic intellect--- a historical, disciplinary, semantic, and probably neurological divide reckoned by the capacitity to reason in both domains--- puts potential builders in a steep quandary. On one hand, the mathematical procession of theorems from a set of symbols and axioms refers to the external world with effective fidelity, but nay absolute certainty. On the other, the capacity to qualitatively reason about the world in all its chance and weirdness resists persistent quantitative evaluation. Why the differentiation in architecture? What constitutes grounds for building? How is building good?

\subsection{Knowledge}
Begin etymologically with \textit{Architecture}: \textit{arkhi-} implying principal, chief, as in \textit{arche}trave, \textit{arche}gonium, \textit{Arch} Linux (a most skeletal Linux distribution); \textit{tekhne} implying art, craft, means, as in geo\textit{techn}ics, \textit{techn}ology, \textit{tec}tonics. The Aristotalean reasoning about ``architecture" as an art in the Nicomachean Ethics is written in the context of ``techne" as one of the chief intellectual virtues.

\begin{quote}
``Now since architecture is an art and is essentially a reasoned state of capacity to make, and there is neither any art that is not such a state nor any such state that is not an art, \textit{art} is identical with a state of capacity to make, involving true reasoning. All art is concerned with coming into being i.e. with contriving and considering how something may come into being, i.e. with contriving and considering how something may come into being which is capable of either being or not being, and whose origin is in the maker and not in the thing made; for art is concerned neither with things that are, or come into being, by necessity, nor with things that do so in accordance with nature (since these have their origins in themselves). Making and acting are different, art must be a matter of making, not of acting. And in a sense chance and art are concerned with the same objects; as Agathon says, `Art loves chance and chance loves art'. Art then, as has been said, is a state concerned with making, involving true reasoning, and lack of art on the contrary is a state concerned with making, involving a false course of reasoning; both are concerned with the variable." \cite[p105]{NICOMACHEANETHICS}
\end{quote}

Mere observation of the etymological roots of ``architecture" shows that architecture, by definition, concerns a capacity to select (being and not being) and configure (consider and contrive) variables through true reasoning, but without explicity concerning scientific knowledge about these variables (not concerning nature or necessity) \textit{nor} empirical means of creation (making not acting). Scientific knowledge is reserved for the epistemic virtue of intellect, and practical means for the phronetic.

\textit{Engineering}: from \textit{engin} or \textit{ingenium}, implying cleverness, stratagem; or implying ability, innate qualities, inborn characteristics. Reliant on applying \textit{science}: from \textit{scire}, knowing, classifying, but as an intellectual virtue, without action.

\begin{quote}
``Scientific knowledge is, then, a state of capacity to demonstrate, and has the other limiting characteristics which we specify in the Analytics, for it is when a man believes in a cetain way and the starting-points are known to him that he has scientific knowledge, since if they are not better known to him than the conclusion, he will have his knowledge only incidentally." \cite[p105]{NICOMACHEANETHICS}
\end{quote}











% \subsection{Engineered Architecture}
% United States Patent Application US20090234696A1: ``Engineered Architecture" (EA); invented by Israeli-American architect/engineer Eli Attia and submitted in March of 2009. Per the Detailed Description of the Patent,
%
% \begin{quote}
%   ``A method [A system] of design, fabrication, and construction management, the method [system] comprising: receiving selections concerning a building shape and a building size; and executing instructions stored in memory of [a] computing device. . . determines that a plurality of design components are associated with the selected building shape and the selected building size, and generates a report concerning a building design comprising the determined plurality of design components." \cite{ATTIA2009EA}
% \end{quote}
%
% EA's method challenges the waste and inefficiency involved in the phasing of modern skyscrapers and other large buildings; coordination, communication, design, drawing, material provisioning, staffing, management, et cetera \cite{ATTIA2009EA}. By conceiving a library of structural components capable of composing structures determined by arbitrary constraints, the energy typically spent
%
%
%
% EA was never implemented in its intended form. In 2010, Attia partenered with Google X to realize EA at Google-scale. At the outset, it was predicted that the program could halve costs of building a commercial structure, revolutionize the accelerating need for urban construction globally, and reap around one hundred-twenty billion dollars in revenue for Google annually \cite{GLOBESEA}. By 2011 the patented work was ripped from Attia entirely; development continued minus the heart and brain of EA. As of March 2018, the debased mutation of EA, Flux, has stalled indefinitely \cite{FUCKFLUX}.




% Compared to current practice, EA claims a 44\% reduction in elapsed time from start of design to completion of construction, 52\% reduction of tasked items and services, 80\% earlier procurement of long-lead items, and 30\% reduction in cost and labor of an typical 30,000$m^2$ building.

% Following an invitation to share the idea with Google executives in 2010, the EA project was adopted by X, a research and development subsidiary of Alphabet Inc..  By 2014, Google had shouldered Attia out of the project and continued development of the software as ``Flux" \cite{GLOBESEA}.
%
%
%
%
%
%
% Here the modern building methodology is not intractably complex, it is artificially complicated.
%
% Most modern buildings may be perceived as layered assemblages of industrial products and painstakingly standardized procedures contrived for the rather elementary ends of physiological protection and psychrometric stability \cite[p88]{MOECONVERGENCE}. Historian Thomas Hughes' term ``technological momentum" renders the complicated accumulation of these technological contraptions engineered to fit socially constructed needs and their future reciprocities \cite{TECHNOMOMENTUM}. On the managerial side, principles of Scientific Management--- rationalized administration--- from the early twentieth century, settled into the architectural ethos during fervent reconstructive mode post-World War I \cite{MCLEODTAYLORISM}. Technological momentum, rationalized administration, and bureaucratic, corporated architecture are nailed into place during the Modernist rush following World War II \cite{KUBOCORPORATION}.
%
% Positivisms behind Computer Aided Design (CAD) and Building Information Modeling (BIM) pose  solutions for the design and construction industries through extradimensional modeling: real-time collaborative design in three dimensions, temporal sequencing along the fourth dimension, cost analysis as the fifth dimension, and lifecycle analysis as a sixth dimension \footnote{Richard McPartland, \textit{BIM dimensions - 3D, 4D, 5D, 6D BIM explained.} NBS. 10 July, 2017. \url{https://www.thenbs.com/knowledge/bim-dimensions-3d-4d-5d-6d-bim-explained}}.Ironically, CAD and BIM, deprived of
%
% \begin{quote}
%   ``An architectural agenda for energy ultimately requires a more fluxable program for buildings. By becoming more programmaticly inexact but in exacting ways, buildings characterized by precisely vague typologies and anticipitory functions can best trigger the emergent properties of actual complexity and the creative evolution of duration." \cite[p245]{MOECONVERGENCE}
% \end{quote}
%
%
% \begin{quote}
%   ``. . . analysis that identifies all involved material and energy flows from the formation of raw materials to end of life of the building." \cite[p. 70]{MOEHEA}
% \end{quote}
