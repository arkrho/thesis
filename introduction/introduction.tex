\section{Introduction}

This thesis is an attempt to derive a fluxable design program for rammed-earth construction from what Kiel Moe has contemporarily called ``an architectural agenda for energy."

\begin{quote}
  ``An architectural agenda for energy ultimately requires a more fluxable program for buildings. By becoming more programmaticly inexact but in exacting ways, buildings characterized by precisely vague typologies and anticipitory functions can best trigger the emergent properties of actual complexity and the creative evolution of duration." \cite[p245]{MOECONVERGENCE}
\end{quote}

Patent US2009/0234696A1, ``Engineered Architecture", submitted by licensed architect Eli Attia and published on September 17, 2009. Description of Related Art: "Current practice for architectural design, fabrication, and construction for buildings includes various inefficiencies and areas of waste. Such inefficiencies may involve coordination, communication, design, material provisioning, material management, etc. There is, therefore, a need in the art for improved systems and methods for automated design, fabrication, and construction management for buildings."


The role of On the international scale, emissions trading enabled by Article 17 of the Kyoto Protocol reduces exergy to a market game and neglecting the externalities s Paris accord, infrastructure, collectives, physiological, quantum.


Unwarranted junctions and conceptual disjunctions between engineering and architecture are shown to have been a significant factor of extensive infrastructural malpractice in the twentieth century \cite{MOEIM}. Namely, isolating patterns of insulated energy design fell out of nineteenth century equilibrium thermodynamics then manifested in physiologically and environmentally malignant building energy systems. This framework for designing indoor climates runs so deeply in the modern practice and discourse that scaled solutions such as solar energy transformers, turbines, biofuels, atomic energy, and the ideology of energy efficiency fail to recognize the local movements of energy itself.

It is postulated here that


To address this complexity a program is formulated. At the outset, ``program" has an architectural sense and a computational sense. ``Architecture" as well has a physical sense and a virtual sense.







\begin{quote}
  ``. . . analysis that identifies all involved material and energy flows from the formation of raw materials to end of life of the building." \cite[p. 70]{MOEHEA}
\end{quote}

\subsection{Fluxability}
Flux is





\subsection{Form Follows Energy}

% Thus the substantial volume of this work regards the codetermination of analytical method with (re-)emergence of a novel (ancient) building ontology.
% Ontological novelty can be seen in the ``smooth" nature of thermally active materials versus the ``striated" nature of componential, layered HVAC systems. Here (then); monolithic, thermally massive, hygroscopic, low-tech cellular solids work (worked) to form an enclosure's energetic, structural, physiological, and ecological bearings. Material instances include: wood, masonry, and hydronically activated earthen composites such as cob, adobe, and rammed-earth (now: engineered timber and glasses, ceramics, foams, polymers, concrete).



% computation complexity language between eng arch program architecture flows diagrams alexander flux Autotune


% energy language model realize system. isolated system isolated model, open model open system. objects and transformations, encoded. complexity. simulation: good and bad.
%
% The thesis is this: Codification of a particular building ontology is of first-order importance in the complex realizations of building systems in time. Thus, constructive decision-making by integrating over a multiplicity of complex and chaotic fields necessitates parity between analytical and programmatic grammars to realize the energetic agenda. Integrative techniques in design and construction with the rammed-earth building material are searched for \footnote{"Maybe the truth in searching is not having found. -- David Longstreth"} along a set of N bases.
%
%
%
% Convergence (Solidarity) \cite[p9-12]{MOECONVERGENCE} gives reasoning for Integrated Design \cite[Preface]{MOEID}
%
% Fluxable programming enters this work through the doors of thermodynamics and computation. It is able to enter both doors at once because ``programming" belongs to both architecture and computational science. ``Fluxable" is a concept expanded in **SECTION***
%
% The fuzzy-logical premises of ``more programmaticly inexact but in in exacting ways" raises a number of quandaries regarding the nature of complexity and the degree to which design can be effective in harnessing it.
%
%
%
% the smooth ontology of thermally active building strategies eclipses the striated ontology of insulated strategies that have accumulated over the past century
%
% computational design is the rising model fit to interface this new agenda by scale, communication, integration, abstracted complexity, simulation, and abstract similitude between concepts like ``program", ``source", ``object", and ``system". Analysis and its associated assumptions are acknowledged to be a culprit in propogating unstable if not outright toxic energy systems
%
% account of rammed-earth construction as procedural convergence from geoparticles to structurally static yet energetically dynamic material matrix.
%
%
% The architectural agenda supposes a scope from the ``molecular to the territorial" which is a terrific range but also carryies concise implications for earthen design and construction.

\subsection{Computational Modeling in Light of Analytical Modeling}

\subsection{Thermal Inertia and Intrinsic Evaporative Cooling}

\subsection{Form of the Model}
