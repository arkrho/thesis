\section{INTRODUCTION}


``Rammed-earth"/``pis\'e de terre"/``tapia"/``h\=angt\v u" refers to an earthen building material formed by vaguely specific technical process. Physically and conceptually enduring, the material-method has diffused around Earth from original concentrations in (at least) Asia, Africa, and the Middle East. \cite{RAMMEDEARTHHOUSE}

\begin{flushright}
\small{
\textit{``Have we not in Africa and in Spain walls of earth, known as `formacean' walls? From the fact that they are moulded, rather than built, by enclosing earth within a frame of boards, constructed on either side. These walls will last for centuries, are proof against rain, wind, and fire, and are superior in solidity to any quarry stone."}}\\ --- Pliny the Elder. \textit{The Natural History}. Circa 77 A.D.. (tr. Bostock. 1855.)
\end{flushright}

Rammed-earth has sustained its own vaguely specific material composition and construction logistic while adapting to various sociotechnical settings and their design logics. In modern U.S. history, rammed-earth has resurfaced in response to agricultural, economic, or environmental calls for a reliable and fundamental building material.

\begin{flushright}
\small{
\textit{
``Historically, rammed earth has expressed itself as an economical do-it-yourself project for farmers, enthusiasts, and environmentalists. It has also been understood as a way to correct social ills, minimize financial difficulties, and remedy overabundances of labor."}}\\ --- Jennifer Lynn Carpenter. \textit{Dirt Cheap: The Gardendale Experiment and Rammed Earth Home Construction in the United States}. 2010.
\end{flushright}

At last, the Second Law has caught up to rammed-earth building in the U.S.. Contemporary technologies such as cement-stabilized, pre-insulated, and pre-fabricated walls disorganize the historical material-method through contemporary fixations on standardization, insulation, and mass-production \cite{MOECONVERGENCE}. It is hypothesized herein that rammed-earth's historical coupling of local materialization with low-tech construction methodology and heuristical desisn philosophy is the key to its veritably sustaining quality (physically and conceptually enduring for many years). Despite the deeply encoded criteria for safety, sustainability, scalability, and a certain kind of comfort today, losses (appearing as sustainable developments) from the original rammed-earth methodology subtly but significantly challenge its pre-scientific notion of sustainability. The objective herein is a system of design for rammed-earth building that does not yield to the contemporary standard because it is just that, nor does it totally regress to a technologically pessimistic stance, rather, it intends to live with one foot in both domains.

\begin{flushright}
\small{
\textit{
``Contemporary stabilized rammed earth (SRE) draws upon traditional rammed earth (RE) methods and materials, often incorporating reinforcing steel and rigid insulation, enhancing the structural and energy performance of the walls while satisfying building codes. SRE structures are typically engineered by licensed Structural Engineers using the Concrete Building Code or the Masonry Building Code."}} \\ --- Bly Windstorm and Arno Schmidt. \textit{A Report of Contemporary Rammed-Earth Construction and Research in North America}. 2013.
\end{flushright}


% \begin{flushright}
%   \small{
%   \textit{``The history of building construction can be construed as a narrative of the inertia and momentum of two divergent construction logistics. One mode, discussed above, has very minimal historical inertia coupled with great current industrial momentum (the muli-layered assemblies of modernity.) The other has great historical, physical, and thermodynamic inertia that is coupled with minimal industrial momentum in the contemporary building industry/building science industry (more monolithic assemblies and masses). The former follows the short history of the twentieth century ``rationalization" of construction, air-conditioning, factory production, lightweight envelopes, and, more recently, mass customization. The latter is a several-thousand-year history of accumulative knowledge and performance all but forgotten in the interesting yet hubristically selective amnesia of twentiety century architecture."}}\\ --- Kiel Moe. \textit{Convergence}. 2013.
% \end{flushright}
