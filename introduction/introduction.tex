\section{Introduction}


``Rammed-earth"/``pis\'e de terre"/``tapial"/``h\=angt\v u" is a particular earthen building material formed by a particular construction process. Physically and conceptually enduring, the material-method has diffused around the world from original concentrations in (at least) Asia, Africa, and the Middle East. \cite{RAMMEDEARTHHOUSE}

\begin{flushright}
\small{
\textit{``Have we not in Africa and in Spain walls of earth, known as `formacean' walls? From the fact that they are moulded, rather than built, by enclosing earth within a frame of boards, constructed on either side. These walls will last for centuries, are proof against rain, wind, and fire, and are superior in solidity to any cement."}}\\ --- Pliny the Elder. \textit{The Natural History}. Circa 77 A.D.. (tr. Bostock. 1855.)
\end{flushright}

Rammed-earth has sustained its own particular composition, construction logistic, and design logic while also adapting to various sociotechnical contexts.

\begin{flushright}
\small{
\textit{
``Historically, rammed earth has expressed itself as an economical do-it-yourself project for farmers, enthusiasts, and environmentalists. It has also been understood as a way to correct social ills, minimize financial difficulties, and remedy overabundances of labor."}}\\ --- Jennifer Lynn Carpenter. \textit{Dirt Cheap: The Gardendale Experiment and Rammed Earth Home Construction in the United States}. 2010.
\end{flushright}

At last, the Second Law has caught up to rammed-earth building in the U.S.. Contemporary technological contrivances such as cement-stabilized, pre-insulated, and pre-fabricated walls disorganize the historical material-method through contemporary fixations on standardization, insulation, and mass-production \cite{MOECONVERGENCE}. It is hypothesized herein that rammed-earth's historical and particular coupling of local material with low-tech construction methodology is the key to its veritably sustaining quality (physically and conceptually enduring for many years). Despite the modernized  criteria of scalability, safety, and sustainability, contemporary deviations from the original rammed-earth methodology subtly but significantly challenge the sustaining quality.

\begin{flushright}
\small{
\textit{
``Contemporary stabilized rammed earth (SRE) draws upon traditional rammed earth (RE) methods and materials, often incorporating reinforcing steel and rigid insulation, enhancing the structural and energy performance of the walls while satisfying building codes. SRE structures are typically engineered by licensed Structural Engineers using the Concrete Building Code or the Masonry Building Code."}} \\ --- Bly Windstorm and Arno Schmidt. \textit{A Report of Contemporary Rammed-Earth Construction and Research in North America}. 2013.
\end{flushright}

What follows is an attempt to design a system of design for rammed-earth building by considering separately the historical logistic of rammed-earth building and the contemporary logistic of building in general. By drawing these two logistics apart, it is theoretically possible to preserve the ante-technological sustainability of rammed-earth in light of and potentially bootstrapped by the modernized, analytical, and hyper-connected building system.

\begin{flushright}
  \small{
  \textit{``The history of building construction can be construed as a narrative of the inertia and momentum of two divergent construction logistics. One mode, discussed above, has very minimal historical inertia coupled with great current industrial momentum (the muli-layered assemblies of modernity.) The other has great historical, physical, and thermodynamic inertia that is coupled with minimal industrial momentum in the contemporary building industry/building science industry (more monolithic assemblies and masses). The former follows the short history of the twentieth century ``rationalization" of construction, air-conditioning, factory production, lightweight envelopes, and, more recently, mass customization. The latter is a several-thousand-year history of accumulative knowledge and performance all but forgotten in the interesting yet hubristically selective amnesia of twentiety century architecture."}}\\ --- Kiel Moe. \textit{Convergence}. 2013.
\end{flushright}



Aside from historic preservation---for which there is no real positive heritage of rammed-earth building in America to preserve---it is argued that preserving the distinct material-method associated with ``rammed-earth" is the key to preserving its historical ability to sustain. Objectively, contemporary contrivances in industry such as cement-stabilized, pre-fabricated, and pre-insulated rammed-earth walls already appear to shear meaning from rammed-earth in order to satisfy important, but not entirely sustainable layers of the building megastructure.

Contemporary stabilized rammed earth (SRE) draws upon traditional rammed earth (RE) methods and materials, often incorporating reinforcing steel and rigid insulation, enhancing the structural and energy performance of the walls while satisfying building codes. SRE structures are typically engineered by licensed Structural Engineers using the Concrete Building Code or the Masonry Building Code.



% Consider previous expressions of rammed-earth in the modern U.S.:
%
% \begin{flushright}
% \small{
% \textit{
% ``Historically, rammed earth has expressed itself as an economical do-it-yourself project for farmers, enthusiasts, and environmentalists. It has also been understood as a way to correct social ills, minimize financial difficulties, and remedy overabundances of labor. During the Great Depression, these factors came together and pushed the federal government to experiment with the technique, erecting seven rammed earth homes as part of the Resettlement Administration’s Gardendale Homestead north of Birmingham, Alabama. They remained an experiment, as a true federal rammed earth initiative never fully developed."}}\\ --- Jennifer Lynn Carpenter. \textit{Dirt Cheap: The Gardendale Experiment and Rammed Earth Home Construction in the United States}. 2010.
% \end{flushright}
%
% Entering the first world to recognize rammed-earth building while knowing ``sustainability" as such; what follows is an attempt to rationalize the reemergence of the ancient material/method within the megastructure of contemporary building and the megaconception of sustainability. Contemporary contrivances such as cement-stabilized, pre-fabricated, reinforced, pre-treated, and pre-insulated rammed-earth walls defy the semantic of rammed-earth as the particular building material conceived from a particular building method.
%
% Scalability and safety
%
% Sustaining design heuristics are searched for in the technics of anonymous rammed-earth building, brought in light of the standard methodology, and an attempt is made to preserve certain aspects of the former while exploiting inefficiencies of the latter.
