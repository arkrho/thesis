\begin{flushright}
\small{
\textit{
``Because the era before the eighteenth century is mistakenly supposed to have been technically backward, one of its best characteristics has been overlooked: namely, that it was still a mixed technology, a veritable polytechnics, for the characteristic tools, machine-tools, machines, utensils, and utilities it used did not derive solely from its own period and culture, but had been accumulating in great variety for tens of thousands of years. . . The introduction of new inventions like the clock did not necessitate on principle the discarding of any of these older achievements." \cite[p.134]{MYTHMACHINE}}}
\end{flushright}

\begin{flushright}
  \small{
  \textit{``. . . the history of rammed-earth and the contemporary
experience of the experimenters will hold a great value to
the builder until the material enjoys the same commonplace
security of the clay-fired brick, a building unit which no
amateur questions but which is far more vulnerable to
faulty manufacture and inexpert handling than rammed-earth
could ever possibly be." \cite[p.xvi]{RAMMEDEARTHHOUSE}}}
\end{flushright}

\begin{flushright}
  \small{
  \textit{``We are searching for some kind of harmony between two intangibles; a form which we have not yet designed and a context which we cannot properly describe." \cite[p.26]{SYNOFFORM}}}
\end{flushright}

``Rammed-earth"/``pis\'e de terre"/``h\=angt\v u" is an earthen building material generated (and semantically inseparable) from a particular mechanical process. It is archaic, sourced globally, and reappearing around modern technological/architectural domains by virtue of its ecological, physiological, and energetic benevolence. This thesis builds on a particular epistemic and technical ground for designing rammed-earth systems with regards to ancient building methodology in light of the inertia of high-technology solutions to modern building and energy quandaries.  Given rammed-earth's general disappearance from modernity, at least in North America, on what grounds is it reappearing? Is it being regrounded by modern principles of science and building, and does this imply forgetting a veritable polytechnic heritage? For that matter, is it necessarily reserved for an isolated sector of knowledge, or may it also appear for a wider collective? These questions turn out to be reflective of the material, the method, and the modern way of building.

Part I deliberately follows a trajectory of building philosophies from eighteenth-century industrialism through twentieth-century modernism and into the contemporary mode, in order to rationalize rammed-earth's own ancient history through the refraction of modernism. By confronting pithy principles of these three eras, it is argued that the efficacy of contemporary rammed-earth design relies on a conscious balance between science and craft, high-tech and low-tech, and computation and intuition.

Part II considers a modern physical law about architecture and modern architectural theory about physics that may together extend to a potential ground for designing and building rammed-earth systems.

Part III summarizes a potential implementation of this ground as a software-based design system.

\clearpage

\section{Function Follows Form Follows Function Follows Energy: Iron to Silicon, Steam to Bits}

``Form follows function" is a well-known principle of modern design, uttered by an architect. ``Function follows form" is an alternative to this principle in a modernizing era, uttered retrospectively by a structural designer. ``Form follows energy" is a contemporary iteration of the principle, uttered by a building scientist. Logically, ``X follows Y" functions as a causal trace through the complex knot of building towards ground for building one way amidst the infinitude of other ways. According to the O.E.D, form is ``The visible shape or configuration of something." Function is ``An activity that is natural to or the purpose of a person or thing." Energy is ``The property of matter and radiation which is manifest as a capacity to perform work." These trinomials are proposed to massively generalize the complexity of three eras in building history: the modernizing, the modern, and the contemporary. They are employed in search for a similar principle that would fit rammed-earth in the contemporary, i.e. ``rammed-earth design follows Y."

Metamorphic boundaries between `engineering' or `designing' forms (as verbs) with regard to reasoning `scientifically' or `artistically' (as adverbs)---philosophical, historical, technical, semantic, or neurological boundaries probably reckoned by the capacitity to reason in both domains---puts potential designers in a steep quandary. On one hand, applying mathematics and physics to modeled systems enables forms and formalities capable of achieving desired states with relatively high fidelity and invariance. On the other hand, the capacity to intuit the world in all of its chance, variability, and quality also permits an ability to bring about form, appealing uniquely to semi-unquantifiable design factors such as the historical, visual, spatial, or somatosensory.

Before looking back, it is important to note that the lines of building history cross with contruction history, technological history, a number of other theoretical histories, and an infinitude of real histories. Theorizing about history seems only tangentially relevant to contemporary technological design, and moreso but seemingly diminishingly so to contemporary architectural design. At an even higher level, authors of building history are usually also engineers and architects, with doctrinal responsibilities to write about it one way or another. \cite[p14]{CONSHISTORY} Here, consulting history to presuppose a ground for building rammed-earth systems is directed towards the evolution of building concepts and causalities through ideological and technical lines.

\clearpage

\subsection{Function Follows Form}

\begin{quote}
  ``The first principle of structural art is that {\large{\textbf{the form controls the forces.}}} In general terms, this means that {\large{\textbf{function follows form}}}" \cite[p87]{TOWERANDBRIDGE}
\end{quote}

Designer, engineer, historian of technology, and professor David Billington provides an account of ``structural art" in \textit{The Tower and the Bridge}.  According to Billington, structural art is a form of building tall structures, bridges, towers, long-span roofs, and similar objects in this class, originating in eighteenth-century Britain. A first epoch of structral engineering began with the industrialization of iron and this phenomenon's reflexive imperative to build more robust infrastructure, and concluded around the ironed construction of the Eiffel Tower as the explosion of steel and reinforced concrete overran the iron world. A second epoch of structural art began in the late nineteenth century and concluded whenabouts W.W.I. appeared on the horizon and building stalled. \cite[p7]{TOWERANDBRIDGE}

Billington concisely places structural art among a disparate but parallel collective of subjects and objects. Although centered around structural art, this view explicity distinguishes between the technological and the scientific, structural design and architectural design, and the structures and machines themselves. These distinctions contribute to a reading of ``function follows form" and a critical stretch of building history.

\begin{enumerate}
  \item [] \textbf{Engineering and Science:} ``There is a fundamental difference between science and technology. Engineering or technology is the making of things that did not previously exist, whereas science is the discovering of things that have long existed. Technological results are forms that exist only because people want to make them, whereas scientific results are fomulations of what exists independently of human intentions. Technology deals with the artificial, science with the natural."\cite[p9]{TOWERANDBRIDGE}

  \item [] \textbf{Structures and Architecture:} ``Structural designers give the form to objects that are of relatively large scale and of single use, and these designers see forms as the means of controlling the forces of nature to be resisted. Architectural designers, on the other hand, give form to objects that are of relatively small scale and of complex human use, and these designers see forms as the means of controlling the spaces to be used by people." \cite[14]{TOWERANDBRIDGE}

  \item[] \textbf{Structures and Machines:} ``As intimitely connected as they are, structures and machines must function differently, they come into being by different social means, and they symbolize two distinctly different types of designs. Structures must not move perceptibly, are custom-built for one specific locale, and are typically designed by one individual. Machines, on the other hand, only work when they move, are made to be used widely, and are in the late twentieth century typically designed by teams of engineers. General statements about technology are frequently meaningless unless this basic distinction is made."\cite[p13]{TOWERANDBRIDGE}

\end{enumerate}

The first distinction between science and engineering/technology has a basis in classical philosophy. Aristotle distinguishes between \textit{episteme} and \textit{techne} in \textit{The Nicomachean Ethics}; respectively they are ``demonstrative knowledge of the necessary and the eternal" and ``knowledge of how to make things." \cite[p104, 105]{NICOMACHEANETHICS} From these capacities for pure knowledge comes the ability to act deliberately in one mode or another. Science would then follow as an application of reasoning about eternal physical laws. Similarly, \textit{techn}ology/archi\textit{tec}ture follow as applications of deliberations about making.

The second distinction between the role of structural designers and architectural designers concerns \textbf{scale} and \textbf{system boundaries}. That is, the act of making structural art is determined by the large-scale structural function as its sole use case. The act of making architecture is here about the human-scale and human-scale activities.

The third distinction between structures and machines is the most dividing as ``they symbolize two distinctly different types of designs." This is tied to the second distinction and they are both far more subjective than the first distinction. The latter two are quite subject to change.


rationalize these boundaries emerging through lens of structural art, Tom F. Peters' \textit{``How the introduction of iron in construction changed and developed thought patterns in design"} is broadly referenced here as a source for epistemological and technical patterns moving through this era. Three modes of thought: `overlay' thought \cite[36]{IRON} of pre-modern design, `model' thought \cite[37]{IRON} developing under the influence of the Enlightenment, and `kit-of-parts' thought \cite[53]{IRON} leading into the twentieth century, generally discern the pre-modern to modern transition in building.

\textit{Overlay thought} appears between the ``veritable polytechnics" of pre-eighteenth century thought and the nucleation of a more structured, analytical form of polytechnics. It is a pre-theoretical idea, relying on a collective of experiential and experimental forms rather than conscious analysis of structural behavior. `Overlay' implies the superposition of invented and analogous building forms and methods, directed towards the progression of grander and more capable structures. The absence of conscious analysis allows ``more flexible and redundant" designs to fit a particular materiality, location, and construction method; compared to calculated, deterministic, and isolated designs derived from an abstract model. \cite[p36, 37]{IRON} In turn, it lacks a robust arithmetic language, but embraces a geometric one.

\textit{Model thought}, \textbf{the generally accepted form of contemporary building thought} \cite[p38]{IRON}, ``[makes] objects by synthesizing analytical thinking and a clear split of empirical knowledge into the experiential (learning from direct knowledge, which [is] considered a lesser process), and the experimental, or learning by introducing a variable into a normal situation, which was considered a more reliable way to gain knowledge." \cite[p39]{IRON} By deconstructing experimentally validated forms known to work reliably for a certain purpose into a hierarchy of parts susceptible to conscious analytical functions, builders are able to solve various problems with rigorous \footnote{De Toffoli claims that sentential logic is regarded in the modernizing traditions as the proper means towards mathematical proof, as compared to diagrammatic/visual logic.} and standardized means \textit{if they assume a host of non-realistic conditions in the analysis.}

\textit{Kit-of-parts thinking} appeals to a more dynamic mode of building inspired by the building process rather than end form. It is proposed to be, in part, a reaction to the mobilization of war-machines at the onset of W.W.I.. ``Assembly" is a critical notion that enables a ``design-matrix of structural constants and variables" capable of organizing finite \textit{systems} and their corresponding forms, mechanics, and functions. Peters implies a semantic separation from construction (``putting together", creating, changing, and manipulating interfaces and connections by altering components") to assembly (``one degree removed from manufacture, fitting together without alteration, minus adaptation") \cite[p53]{IRON}




The technological innovations driving structural art followed from ``stretching" iron, steel, and concrete beyond successive limits, ``just as medieval designers had stretched stone into the skeletal Gothic cathedral." \cite[p5]{TOWERANDBRIDGE} Doing so enabled a rapid and reciprocating techno-social progress. However, this elasticism was not actually calculable until Claude-Louis Navier's theory of elasticity in the 1820's. \cite[p73]{RETROFITTINGMASONRY} Therefore, Billington claims that the engineering involved in structural art initially denied stress-related analyses for physical intuition and empiricism as its primary science. \cite[p43]{TOWERANDBRIDGE} In light of Mumford's conception of true polytechnics, the structural artists of the eighteenth century polytechnical school appear to be \textbf{the entrance for analytical rigor in building design.}


To handle the ethical pressures of government, shareholders, and industrialists idealizing the conservation of natural and public resources, in addition to the physical demands of building, three disciplines of structural art emerged to cover three dimensions of structure proposed by Billington. ``Efficiency, economy, and elegance" address the scientific dimension, the social dimension, and the symbolic dimension. \cite[p5,16,17]{TOWERANDBRIDGE}




It is conclusive from these propositions of the structural art that science and engineering are separate but parallel disciplines, similarly, structures and machines are separate but parallel objects, and structural design and architecture are separate but parallel disciplines as well. If the conversation between engineering and science is marked by conceptions of the artificial (variable) and conceptions of the natural (invariable), then a rift appears along the \textbf{system boundaries} of a model. System boundaries conceptually determine which quantities are of analytical concern and which are to be assumed external to the function of the system. If the conversation between structural design and architecture is marked by a physical object's magnitude and use, then a rift appears at \textbf{scaling}.

% \begin{quote}
%   ``The prototypical engineering form---the public bridge---requires no architect. The prototypical architectural form---the private house--- requires no engineer. We have seen that scientists and engineers develop their ideas in parallel and sometimes with much mutual discussion; and that engineers of structure must rely on engineers of machinery just to get their works built. Similarly, structural engineers and architects learn from each other and sometimes collaborate fruitfully, especially when, as with tall buildings, large scale goes together with complex use. But the two types of designers act predominantly in different spheres." \cite[p14]{TOWERANDBRIDGE}
% \end{quote}
%
% \begin{flushright}
% \small{
% \textit{``In his 1957 book \textrm{The Reason of Structural Types}, Torroja wrote that ``Vain would be the enterprise of somebody who would propose himself to design a structure without having understood to the backbone the mechanical principles of inner equilibrium." Being an engineer, he saw those principles and their interiorization as something that was independent from history. I would like to challenge that position. For in the understanding of the backbone of the mechanical principles of inner equilibrium we are necessarily indebted to the way we perceive our body and its movements. In the past decades, cultural historians have multiplied sudies showing that this perception is to a certain extent a social construct. Gravity, just like a certain number of fundamental natural constraints, is always perceived through the prism of our body, a historically determined body."}} \\
% --- Antoine Picon \cite[p.23]{CONSHISTORY}
% \end{flushright}



%
