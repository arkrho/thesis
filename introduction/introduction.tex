\section{INTRODUCTION}


``Rammed-earth"/``pis\'e de terre"/``tapia"/``h\=angt\v u" refers to a vaguely specific earthen building material formed by a vaguely specific technical process. Physically and conceptually enduring, the material-method has diffused around Earth from original concentrations in (at least) ancient Asia, Africa, and the Middle East. \cite{RAMMEDEARTHHOUSE}

\begin{flushright}
\small{
\textit{``Have we not in Africa and in Spain walls of earth, known as `formacean' walls? From the fact that they are moulded, rather than built, by enclosing earth within a frame of boards, constructed on either side. These walls will last for centuries, are proof against rain, wind, and fire, and are superior in solidity to any quarried stone."}}\\ --- Pliny the Elder. \\ \textit{The Natural History}. c.a. 77 A.D.
\end{flushright}

Rammed-earth has sustained its own vaguely specific material composition and constructional logistic while adapting to variously similar socio-technical settings throughout the past millennia. For instances, it remains a vernacular form of architecture in rural China, was presented to then-Secretary of State Thomas Jefferson by a French architect in 1789 as ``The economical building art of the ancients, perfected and made more universal \footnote{\url{http://archive.is/yWexi}}," briefly held the attention of the New Deal-era Resettlement Administration as an economical building alternative during the Great Depression \cite{GARDENDALE}, captured marginal interest during the environmental movements of the 1960s and 1970s \cite{GARDENDALE}, and critically, appears in professional architectural design today (with high-profile rammed-earth homes valued in the multimillion-dollar range\footnote{\url{http://archive.is/K853p}}). As rammed-earth design/building toes the waters of the main stream, at least in the U.S., the Second Law is at last catching up to the authentic material-method at its conceptual and technical cores.

\begin{flushright}
\small{
\textit{
``Contemporary stabilized rammed earth (SRE) draws upon traditional rammed earth (RE) methods and materials, often incorporating reinforcing steel and rigid insulation, enhancing the structural and energy performance of the walls while satisfying building codes. SRE structures are typically engineered by licensed Structural Engineers using the Concrete Building Code or the Masonry Building Code."}} \\ --- Bly Windstorm and Arno Schmidt. \\ \textit{A Report of Contemporary Rammed-Earth Construction and Research in North America}. 2013.
\end{flushright}

Contemporary technologies such as cement-stabilized, pre-insulated, and pre-fabricated walls---innovations currently driven by standard fixations on sustainable development, insulation, energy efficiency, and mass-production---disorder the typical, historical coherence of the material-method \cite{MOECONVERGENCE}. It is hypothesized herein that this coherence of locally-sourced materialization with low-tech building methodology is the key to rammed-earth's veritably sustaining quality (physically and conceptually enduring for many years), contributing to the contemporary notion of sustainability within the context of building energy's major role in the global Sun-Earth-Space thermodynamic system.

\begin{flushright}
  \small{
  \textit{``The history of building construction can be construed as a narrative of the inertia and momentum of two divergent construction logistics. One mode, discussed above, has very minimal historical inertia coupled with great current industrial momentum (the muli-layered assemblies of modernity.) The other has great historical, physical, and thermodynamic inertia that is coupled with minimal industrial momentum in the contemporary building industry/building science industry (more monolithic assemblies and masses). The former follows the short history of the twentieth century ``rationalization" of construction, air-conditioning, factory production, lightweight envelopes, and, more recently, mass customization. The latter is a several-thousand-year history of accumulative knowledge and performance all but forgotten in the interesting yet hubristically selective amnesia of twentiety century architecture."}}\\ --- Kiel Moe. \\ \textit{Convergence}. 2013.
\end{flushright}

\clearpage

What follows is an attempt to demonstrate the above hypothesis [the coherence (several-thousand-year history of accumulative knowledge and performance) of the ``traditional" rammed-earth material-method, preserved against an increase in ontological entropy (the multi-layered assemblies of modernity), engenders contemporary rammed-earth building with an ecologically deep, time-tested quality of sustainability that is obliquely approached by modern building practices] as an abstract design system manifest as software. 
