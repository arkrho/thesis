\section{BUILT EARTH}

The historic, monolithic, massive inertia of rammed-earth meets the momentum of modern, multi-layered, lightweight building industry. Upon this meeeting, basic physics suggests an interaction between the two bodies, however, it is not immediately apparent as to how rammed-earth's historically sustaining qualities interface with the conception of sustainable development, nor how the relatively laissez-faire rammed-earth technic fits the complicated structuring of contemporary building.

\subsection{Sustainable Development as Such}
If rammed-earth reappeared in the United States in the 1840s to fit a growing rural economy, and reappeared around the Great Depression to remedy socioeconomic woes \cite{RAMMEDEARTHHOUSE}, there can be little doubt that rammed-earth appears today in regards to sustainable development. For the sake of argument, sustainable development/sustainability as such was mainly conceived in the 1960s, when adverse anthropogenic effects could, for the first time, be observed and measured globally. \cite{ORIGINSOFSUS} The U.N.'s first major deliberation on environmental issues, The Stockholm Conference (1972), ``marked a turning point in the development of international environmental politics." \footnote{sustainabledevelopment.un.org/milestones/humanenvironment} The global conversation about humanity and the environment indeed turned political, and

Discursive method forwarded by the Brundtland Repor ---> recursive, collaborative standardization



\subsection{Materials and Products}
% The multi-layered physical assemblies of modern building characterize a grander conceptual layering associated with modern building, energy, people, and the environment. Professor Moe cogently expressed this idea in \textit{Insulating Modernism}. In the twentieth century, there was a ``bifurcation from reality" wherein isolated, reductive applications of heat transfer and thermodynamics (stemming from refrigeration technology)  coupled with ``mounting market pressure"
%
%
% For instance, ``the International Code Council's (ICC's) International Green Construction code (IgCC) is an overlay code, meaning it is written in a manner to be used with all the other ICC codes" \footnote{\url{https://www.energycodes.gov/development/green/codes}}. For an additional cost of \$38.00, ASHRAE Standard 189.1: ``Standard for the Design of High-Performance Green Buildings" is available as a version containing all additions, subtractions, and alterations since its publication but four years ago.



\subsection{Veritable Sustainability?}
