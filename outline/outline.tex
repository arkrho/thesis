\section{Outline}

INTRODUCTION:

\begin{outline}

\1 GeoEcoHydroSphere Mapping
\1 Material Matrix
\1 Form Follows Energy


\1 Attia's Engineered Architecture
  \2 Information



\1 Engineering and Architecture
  \2 Historical Gaps
  \2 BIM, Convergence, and Limitations
    \3 Parametrics, Hierarchy, Without Feedback or Nonlinearity
\1 Building Systems
  \2 Energy as Common Language
  \2 Format and Structure of Energy Language
\1 Rammed-Earth
  \2 Import Standard Library
  \2 Loss of Computation
  \2 Place in the Modern Building Ethos
\1 Fluxability
  \2 Exacting Inexactly
  \2 Actual Complexity
  \2 The Creative Evolution of Duration
\end{outline}












\begin{outline}
\1 Bases of the Program
  \2 Physiological Basis
    \3 Radiation Nation
    \3 Humidity City
  \2 Energetic Basis
    \ Maximum Power Design
    \ Energy Systems Language
  \2 Computational Basis
    \3 Fluxable Program
      \4 Integrated Design
        \4 Flux.io
  \2 Ecological Basis
    \3 Deep Ecology
    \3 Cascades of Complexity
    \3 Exergy per Emergy
    \3 Exacting Program
      \4 The City is Not a Tree
      \4 AutoTune
      \4 ASHRAE Guideline 14-2002



    \2 Import Rammed-Earth in the context of TAS
      \3 radiant heat transfer a first-order design line
      \3 intrinsic moisture handling is fine
      \3 exergy per emergy
      \3 cascades of benevolence
    \2 Tuning Structure to Environment
      \3 striated complexity for smooth complexity
      \3 how are we to model TAS? (RE)
      \3 scope and scale of the model
    \3


\end{outline}


For Alexander, this urban system is like a semi-lattice in set theory. Two sets of objects and activities overlap at the newsrack. If diagrammed like a branching structure, the branches overlap and connect. The semi-lattice diagrams “natural” cities like Siena, Liverpool, Kyoto, or Manhattan. The tree segregates urban functions in an organization, while the semi-lattice offers “ambiguity” and “multiplicity” in a structure that is “thick, tougher, more subtle, and more complex.” On the one hand, Alexander expands the repertoire of design to include activity. But on the other, he quickly codifies and taxonomizes that activity. He mimics the object of this own critique by reforming the artificial with a “natural” corrective—instead of the tree, the semi-lattice becomes the placeholder. Despite his attempt to incorporate active form and information, Alexander only creates another immobilized form.8
