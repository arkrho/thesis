Rammed-earth is appearing today (at least in the U.S.) through academic programs (theoretical and in-situ), technological research (structural and thermal), and professional architectural design (modest and opulent). Considering the periodic resurgences of rammed-earth throughout modernity \cite{RAMMEDEARTHHOUSE}, perhaps it is not so peculiar that it should resurface again.

\begin{flushright}
\small{
\textit{
``The current level of interest is actually the third wave of rammed earth popularity in the past two hundred years. The first wave reached its high point during the 1840s, and the second during the Great Depression of the 1930s. Both of the first two surges were stimulated by the quest for a low-cost, owner-capable method of construction; and both ended when mass-production and low-cost transportation made other, faster construction methods more accessible"}} \cite[p11, 12]{RAMMEDEARTHHOUSE}
\end{flushright}

Particular geographic and cultural domains---for instance, Asia, the Middle East, and North Africa---share a veritably ancient heritage of earthen building, generally, and rammed-earth building, specifically. \cite{RAMMEDEARTHHOUSE} To contextualize this ancient form of building as fundamentally remote from contemporary building, Christopher Alexander discerns those building cultures wherein architectural, technological, and artistic forms are produced without reference to architecture, technology, or art as such:

\begin{flushright}
\small{
\textit{
``The cultures I choose to call ``unselfconscious" have, in the past, been called by many other names --- each name chosen to illuminate whatever aspect of the contrast between kinds of culture the writer was most anxious to bring out. Thus they have been called ``primitive," to distinguish them from those where kinship plays a less important part in social structure; ``folk," to set them apart from urban cultures; ``closed," to draw attention to the responsibility of the individual in today's more open situation; ``anonymous," to distinguish them from cultures in which a profession called ``architecture" exists.}} \cite[p33]{SYNOFFORM}
\end{flushright}

Alexander acknowledges the evolution of communication technology, conceptualizations (science of building), specialization, and growths in innovation and economy \cite[p33,34]{SYNOFFORM} as factors of the exponential increase in building's functional complexity (both in individual components and collective systems) over these thousands of years.

Given the immense distance between our scientifically and technologically advanced building methodologies and the innate earthen building methodologies, on what ground does the archaic simplicity of rammed-earth fit the modern concept of sustainability? This thesis argues that contemporary rammed-earth design toward the ante-modern image of sustainability implies design systems that regard the ``veritable polytechnics" of ante-modern sustainable design; paradoxically, wherein ``sustainable design" has not been practices as such.

\begin{flushright}
\small{
\textit{
``Because the era before the eighteenth century is mistakenly supposed to have been technically backward, one of its best characteristics has been overlooked: namely, that it was still a mixed technology, a veritable polytechnics, for the characteristic tools, machine-tools, machines, utensils, and utilities it used did not derive solely from its own period and culture, but had been accumulating in great variety for tens of thousands of years. . . The introduction of new inventions like the clock did not necessitate on principle the discarding of any of these older achievements." \cite[p.134]{PENTAGON}}}
\end{flushright}

\clearpage

Into a later---more technically deliberate context---Pliny writes, about a century after Vitruvius' publication of \textit{De Architectura}:

\begin{flushright}
\small{
\textit{``Have we not in Africa and in Spain walls of earth, known as `formacean' walls? From the fact that they are moulded, rather than built, by enclosing earth within a frame of boards, constructed on either side. These walls will last for centuries, are proof against rain, wind, and fire, and are superior in solidity to any cement."}} \cite[p385]{NATHISTORY}
\end{flushright}

Into a moderninzing context, Thomas Jefferson writes to William Short concerning French architect Francois Cointeraux's advocacy for rammed-earth in the United States:

\begin{flushright}
\small{
\textit{``I had seen buildings in this way near Lyons, and moreover had known the author at Paris, where he raised some walls to shew his manner: and afterwards, while I was secretary of state, the President recieved from him lengthy details \& propositions on the same subject. How far it may offer benefit here superior to the methods of the country, founded in the actual circumstances of the country as to the combined costs of labour \& materials, and the circumstances of durability comfort \& appearance, must be the result of calculation."}} \footnote{“To Thomas Jefferson from Francois Cointeraux, 16 June 1789,” Founders Online, National Archives, last modified February 1, 2018, \url{http://founders.archives.gov/documents/Jefferson/01-15-02-0192}. [Original source: The Papers of Thomas Jefferson, vol. 15, 27 March 1789 – 30 November 1789, ed. Julian P. Boyd. Princeton: Princeton University Press, 1958, pp. 184–186.]}
\end{flushright}

\clearpage

Part I: FRAMEWORK FOR FORMWORK
  1. Semantics
    a. Energy / Entropy Crisis
    b. Sustainability / Efficiency
    c. Engineering / Architecture
    d. Open-Source Architecture and Computation


Part II: THEORETICAL GROUNDS: a physical theory of architecture and an architectural theory of physics

Part III: APPLICATION in THREE HEURISTICS: mapping sources of matter and energy, designing, and composing.
