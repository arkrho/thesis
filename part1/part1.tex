Rammed-earth appears today mainly within the context of sustainable development. Paradoxically, sustainability as such followed centuries after the archaeological roots of rammed-earth architecture took ground \cite{RAMMEDEARTHHOUSE}, and probably centuries before ``architecture" was conceived as such.

\begin{flushright}
\small{
\textit{
``The cultures I choose to call ``unselfconscious" have, in the past, been called by many other names --- each name chosen to illuminate whatever aspect of the contrast between kinds of culture the writer was most anxious to bring out. Thus they have been called ``primitive," to distinguish them from those where kinship plays a less important part in social structure; ``folk," to set them apart from urban cultures; ``closed," to draw attention to the responsibility of the individual in today's more open situation; ``anonymous," to distinguish them from cultures in which a profession called ``architecture" exists.}} \cite[p33]{SYNOFFORM}
\end{flushright}

There appears to be an irreversible epistemic gap between rammed-earth's original tectonic and contemporary systems of design applying to the rammed-earth material. It is argued that the intrinsic coupling between material and method heard in ``rammed-earth"/``pis\'e de terre"/``h\=angt\v u" has sustained the rammed-earth methodology across worlds of context. Pliny writes, about a century after Vitruvius published his treatise on architecture:

\begin{flushright}
\small{
\textit{``Have we not in Africa and in Spain walls of earth, known as `formacean' walls? From the fact that they are moulded, rather than built, by enclosing earth within a frame of boards, constructed on either side. These walls will last for centuries, are proof against rain, wind, and fire, and are superior in solidity to any cement."}} \cite[p385]{NATHISTORY}
\end{flushright}

Into modernity, Thomas Jefferson writes to William Short regarding French architect Francois Cointereaux's advocacies for rammed-earth building with American soil:

\begin{flushright}
\small{
\textit{``I had seen buildings in this way near Lyons, and moreover had known the author at Paris, where he raised some walls to shew his manner: and afterwards, while I was secretary of state, the President recieved from him lengthy details \& propositions on the same subject. How far it may offer benefit here superior to the methods of the country, founded in the actual circumstances of the country as to the combined costs of labour \& materials, and the circumstances of durability comfort \& appearance, must be the result of calculation."}} \footnote{``I. To William Short, 13 April 1800,” Founders Online, National Archives, last modified February 1, 2018, http://founders.archives.gov/documents/Jefferson/01-31-02-0432-0002. [Original source: The Papers of Thomas Jefferson, vol. 31, 1 February 1799 – 31 May 1800, ed. Barbara B. Oberg. Princeton: Princeton University Press, 2004, pp. 501–511.]}
\end{flushright}

Consider the context of rammed-earth resurgences leading into the twenty-first century:

\begin{flushright}
\small{
\textit{
``Historically, rammed earth has expressed itself as an economical do-it-yourself project for farmers, enthusiasts, and environmentalists. It has also been understood as a way to correct social ills, minimize financial difficulties, and remedy overabundances of labor. During the Great Depression, these factors came together and pushed the federal government to experiment with the technique, erecting seven rammed earth homes as part of the Resettlement Administration’s Gardendale Homestead north of Birmingham, Alabama. They remained an experiment, as a true federal rammed earth initiative never fully developed."}}\cite{GARDENDALE}
\end{flushright}

% In this narrow frame, rammed-earth fit the nineteenth century rural economy as a durable, self-built, agricultural building material, and it fit twentieth century socio-economic distress as a practical form of housing. What succeeded  forms a ground for its present resurgence.
%
% Particular geographic and cultural domains---for instance, Asia, the Middle East, and North Africa---share the ancient heritage of earthen building, generally, and rammed-earth building, specifically. \cite{RAMMEDEARTHHOUSE} To define the ancient building culture as fundamentally distant from contemporary building practice, Christopher Alexander discerns those building cultures wherein architectural, technological, and artistic forms are produced without reference to architecture, technology, or art as such:
%
%
% Alexander acknowledges the evolution of communication technology, conceptual modeling (science of building), specialization, and growth of innovation and economy \cite[p33,34]{SYNOFFORM} as factors of the exponential increase in building's functional complexity (both in individual components and collective systems) over these thousands of years. Despite appropriation by a world of cultures in varying stages of human development, the rammed-earth material and method have remained consistent, perhaps through a linguistic coupling of material and method in, for instance,  ``rammed-earth"/``pis\'e de terre"/``h\=angt\v u/``formacean."
%
% Coming into the classical, as a more technically deliberate era than the ancient, Pliny writes (about a century after Vitruvius' original treatise on architecture as architecture):
%
%
% Into the moderninzing context, Thomas Jefferson writes to William Short regarding French architect Francois Cointeraux's advocacy for agricultural rammed-earth structures in the U.S.:
%
%
% What follows is an attempt to rationalize and synthesize a heuristical, computational design system for rationalizing and synthesizing rammed-earth systems. Preserving the sustaining qualities of rammed-earth building, ground in its material and methodological constancy, forms the basis for this design system.


% Paradoxically, ``sustainability" does not appear in the mainstream vernacular until the 1960s. As such, today's building complex is quite seriously recognizing the archaism that is rammed-earth building---one that thrived without an explicit consciousness for sustainable design---motivated by paradigms not totally apparent or formalized until the mid-twentieth century. If actually paradoxical, does this phenomenon (shadowed by the size and inertia of technological progress) carry technical and/or epistemic implications for building sustainability and the contemporary design of rammed-earth systems?

% This is to say that rammed-earth's veritably sustainable quality, in sustaining itself materially and methodologically for centuries, is a function of its archaic roots.

% It appears today (at least in the U.S.) through academic programming (theoretical and in-situ), technological research (structural and thermal), and professional architecture (modest and sophisticated).

\clearpage

RATIONALIZING A FRAMEWORK FOR FORMWORK
  1. Sustainability
    a. Ecological Economics
  2. Thermodynamics
  3. Computation

SYNTHESIZING A FORMWORK FOR RAMMED-EARTH DESIGN

Part II: THEORETICAL GROUNDS: a physical theory of architecture and an architectural theory of physics

Part III: APPLICATION in THREE HEURISTICS: mapping sources of matter and energy, designing, and composing.
