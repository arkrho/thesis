\begin{flushright}
  \small{
  \textit{``. . . the history of rammed-earth and the contemporary
experience of the experimenters will hold a great value to
the builder until the material enjoys the same commonplace
security of the clay-fired brick, a building unit which no
amateur questions but which is far more vulnerable to
faulty manufacture and inexpert handling than rammed-earth
could ever possibly be."} \\
--- Anthony F. Merrill, \textit{The Rammed-Earth House}. 1947.}
\end{flushright}

The rammed-earth building material has been gathering a contemporary wave of modern scientific, technological, and architectural inspiration globally. Particular geographic domains, for instance: China, the Middle East, and Africa, have veritably ancient archaeological histories in earthen building (generally) and rammed-earth building (specifically). More technically deliberate rammed-earth structures are known to follow in Spain, France, Italy, and the Native Americas. In modern America, Rick Joy Architects (Tuscon, AZ) and Rammed Earth Works (Napa, CA) designed professional, high-value, prize-winning rammed-earth homes, and contemporary rammed-earth is ``experiencing a renaissance in Australia that is unparalleled anywhere else in the world." \cite{RAMMEDEARTHHOUSE}

\begin{flushright}
  \small{
  \textit{
``As public awareness grows regarding the importance of environmentally responsible building methods, rammed earth in the United States is gaining much-deserved recognition. The current level of interest is actually the third wave of rammed earth popularity in the past two hundred years. The first wave reached its high point during the 1840s, and the second during the Great Depression of the 1930s. Both of the first two surges were stimulated by the quest for a low cost, owner-capable method of construction: and both ended when mass-production and low-cost transportation made other, faster construction methods more accessible." } \\
 --- David Easton. \textit{The Rammed Earth House}. 2007.}
\end{flushright}

As such, the ancient form of building finds new life in the infinitely complex cosmology of post-modern building science and technics. From innocent/unselfconscious design, qualities Christopher Alexander attributed to pre-logical problem solving \cite[p8]{SYNOFFORM}, to societies of design based on information networks and proof; on what ground is rammed-earth reappearing? Is it \textit{sustainable}?

It is argued that rammed-earth, as a type of earthen architecture, reappears in the upper echelons of modern science, technology, architecture, and building as being defined by the one word: ``sustainable", within a particular context, namely, postmodern Energy Crisised societies. Given the contemporary discourse on the relation between building and energy, at what scales and within what boundaries are rammed-earth systems sustainable?
